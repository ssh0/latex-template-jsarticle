\documentclass{jsarticle}

% rendeing with dvipdfmx
\usepackage[dvipdfmx]{graphicx}

% floating image
\usepackage{float}

% ams series for math
\usepackage{amsmath}
\usepackage{amssymb}
\usepackage{amsthm}
\usepackage{ascmac}

% use hyperref & 日本語での栞の文字化けを防ぐ
\ifx\kanjiskip\undefined\else
  \usepackage{atbegshi}
  \ifx\ucs\undefined
    \ifnum 42146=\euc"A4A2
      \AtBeginShipoutFirst{\special{pdf:tounicode EUC-UCS2}}
    \else
      \AtBeginShipoutFirst{\special{pdf:tounicode 90ms-RKSJ-UCS2}}
    \fi
  \else
    \AtBeginShipoutFirst{\special{pdf:tounicode UTF8-UCS2}}
  \fi
  \usepackage[dvipdfmx]{hyperref}
\fi

% for code
% \usepackage{moreverb}

% 数式改ページの許可
\allowdisplaybreaks[1]

% 定義,定理,証明を日本語表記に
\theoremstyle{definition}
\newtheorem{theorem}{定理}
\newtheorem*{theorem*}{定理}
\newtheorem{definition}[theorem]{定義}
\newtheorem*{definition*}{定義}
\renewcommand\proofname{\bf 証明}

\hypersetup{breaklinks=true,
            pdfauthor={著者},
            pdftitle={タイトル},
            colorlinks=true,
            citecolor=black,
            urlcolor=black,
            linkcolor=black,
            pdfborder={0 0 0}}
\urlstyle{same}  % don't use monospace font for urls

% "\vector{a}" で太字ベクトル
\def\vector#1{\mbox{\boldmath \(#1\)}}

\title{タイトル}
\author{著者}
\date{\today}

\begin{document}
\maketitle

% % 章・節・図の番号の開始番号を指定(1つ前の数字指定する)。
% \setcounter{section}{5}
% \setcounter{subsection}{2}
% \setcounter{figure}{8}

% % 『第n章』のように出力
% \renewcommand{\presectionname}{第}
% \renewcommand{\postsectionname}{章}

%%%%%%%%%%%%%%%%%%%%    本文    %%%%%%%%%%%%%%%%%%%%

% 図の挿入(その場に出力)
% \begin{figure}[H]
%     \begin{center}
%         \includegraphics[width=0.6\textwidth]{./path/to/some/img/figure_01.pdf}
%         \caption{キャプション}
%         \label{f1}
%     \end{center}
% \end{figure}




%%%%%%%%%%%%%%%%%%%%    本文    %%%%%%%%%%%%%%%%%%%%

% % 参考文献
% \bibliographystyle{junsrt}
% % reference.bib を参照するとき
% \bibliography{reference}

\end{document}
